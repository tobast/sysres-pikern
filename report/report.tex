\documentclass[11pt,a4paper]{article}
\usepackage[utf8]{inputenc}
\usepackage[francais]{babel} %% FRENCH, FIXME if typing in english
\usepackage[T1]{fontenc}
\usepackage{amsmath}
\usepackage{amsfonts}
\usepackage{amssymb}
\usepackage{graphicx}
\usepackage[left=2cm,right=2cm,top=2cm,bottom=2cm]{geometry}

% Custom packages
\usepackage{my_listings}
\usepackage{my_hyperref}
\usepackage{math}

\author{Théophile \textsc{Bastian}, Nathanaël \textsc{Courant}}
\title{Systèmes et réseaux~: rendu de projet\\
{\small PiKern, un noyau minimaliste pour Raspberry Pi}}
\date{28 mai 2016}

\newcommand{\todo}[1]{\colorbox{orange}{\color{blue}{\Large TODO:} #1}}

\begin{document}
\maketitle

\begin{abstract}
Au cours de ce projet, nous nous sommes intéressés à l'écriture en C++ d'un
noyau minimaliste bootable pour Raspberry Pi. Nous avons réussi à implémenter
une gestion de processus distincts avec ordonnanceur, une couche réseau
complète gérant le ping et l'UDP,

	\todo{suite}
\end{abstract}

\tableofcontents
\newpage

\section{Vue d'ensemble}

\subsection{organisation du code}

\section{Bas niveau et matériel}

\section{Processus}

\section{Réseau}

\section{Difficultés rencontrées}

\subsection{Compilation du code}\label{ssec:diff:compil}
\todo{}
\subsection{Recoder la bibliothèque standard}\label{ssec:diff:stdlib}
\todo{}
\subsection{Écrasement de la table de vecteurs}\label{ssec:diff:delvect}
\todo{}
\subsection{Liaison C et C++ à la fois de malloc}\label{ssec:diff:linkage}
\todo{}
\subsection{\lstc{.text.startup} n'est pas au début}\label{ssec:diff:startup}
\todo{}
\subsection{Stack des processus mal placée}\label{ssec:diff:async_stack}
\todo{}

\end{document}

