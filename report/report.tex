\documentclass[11pt,a4paper]{article}
\usepackage[utf8]{inputenc}
\usepackage[francais]{babel} %% FRENCH, FIXME if typing in english
\usepackage[T1]{fontenc}
\usepackage{amsmath}
\usepackage{amsfonts}
\usepackage{amssymb}
\usepackage{graphicx}
\usepackage[left=2cm,right=2cm,top=2cm,bottom=2cm]{geometry}

% Custom packages
\usepackage{my_listings}
\usepackage{my_hyperref}
\usepackage{math}

\author{Théophile \textsc{Bastian}, Nathanaël \textsc{Courant}}
\title{Systèmes et réseaux~: rendu de projet\\
{\small PiKern, un noyau minimaliste pour Raspberry Pi}}
\date{29 mai 2016}

\newcommand{\hex}[1]{\texttt{0$\times$#1}}
\newcommand{\fname}[1]{\texttt{#1}} %% NOTE use this to format file names

\newcommand{\todo}[1]{\colorbox{orange}{\color{blue}{\Large TODO:} #1}}

\begin{document}
\maketitle

\begin{abstract}
Au cours de ce projet, nous nous sommes intéressés à l'écriture en C++ d'un
noyau minimaliste bootable pour Raspberry Pi. Nous avons réussi à implémenter
une gestion de processus distincts avec ordonnanceur, une couche réseau
complète gérant le ping et l'UDP, un système de fichiers en mémoire
pouvant contenir des fichiers exécutables, un shell distant
minimaliste, ainsi que quelques jeux~: un serveur snake, et une
implantation basique de la Z-machine, permettant de jouer à un grand
nombre de jeux d'\textit{interactive fiction}, en particulier une
bonne partie des jeux publiés par Infocom.

	\todo{suite}
\end{abstract}

\begin{center}
	\href{https://github.com/tobast/sysres-pikern}
		{\Large \includegraphics[height=3em]{github.png}
		\raisebox{1em}{\textbf{Code source}}}
\end{center}

\tableofcontents
\newpage

%%%%%%%%%%%%%%%%%%%%%%%%%%%%%%%%%%%%%%%%%%%%%%%%%%%%%%%%%%%%%%%%%%%%%%%%%%%%%%%
\section{Vue d'ensemble}

\todo{Résumer ce qu'on a fait, utilisation d'USPi, etc.}

\subsection{Organisation du code}

Tout le code source du noyau peut être trouvé sur Github
\href{https://github.com/tobast/sysres-pikern}{ici}, versionné.

Le code du noyau (dossier \fname{kernel/src}) est organisé dans les fichiers
suivants~:

\begin{itemize}
	\item \fname{arp}~: gestion des paquets ARP (découverte d'adresse MAC)~;
	\item \fname{assert}~: gestion des assertions~;
	\item \fname{atomic}~: opérations atomiques (mutex, \ldots)~;
	\item \fname{barriers}~: barrières mémoire et d'instructions~;
	\item \fname{Bytes}~: classe munie d'opérateurs \lstc{<<} et \lstc{>>} pour gérer des
		octets en big endian, utilisé essentiellement pour le réseau~;
	\item \fname{cinit}~: fonction sur laquelle saute le point d'entrée
		(écrit en assembleur) après avoir déplacé le stack pointer~;
	\item \fname{common}~: déclarations de types et de fonctions usuelles~;
	\item \fname{exec\_context}~: contexte d'exécution d'un programme
		(stdin, stdout, working directory, \ldots)~;
	\item \fname{expArray}~: équivalent à \lstcpp{std::vector}~;
	\item \fname{filesystem}~: un RAMFS~;
	\item \fname{format}~: formatteur sur le modèle de sprintf, pour des
		\fname{Bytes}~;
	\item \fname{fs\_populator}~: remplissage initial du RAMFS~;
	\item \fname{genericSocket}~: classe de base pour les sockets (entre
		applications et réseau)~;
	\item \fname{gpio}~: gestion des GPIOs (pins logiques matériels)~;
	\item \fname{hardware\_constants}~: constantes dépendant de la version
		de Raspberry Pi utilisée~;
	\item \fname{hashTable}~: une table de hachage~;
	\item \fname{icmp}~: gestion des paquets ICMP (actuellement, uniquement
		réponse au ping)~;
	\item \fname{init.s}~: point d'entrée (assembleur) du programme~;
	\item \fname{interrupts}~: gestion des interruptions~;
	\item \fname{ipv4}~: gestion de la couche IPv4~;
	\item \fname{kernel}~: fonction principale (main)~;
	\item \fname{logger}~: interface de logs via le réseau~;
	\item \fname{mailbox}~: gestion du protocole \textit{mailbox}, permettant
		de dialoguer avec le matériel (récupération de l'adresse MAC, 
		de la température des composants, de gérer l'alimentation des
		composants, \ldots)~;
	\item \fname{malloc}~: malloc naïf (\lstc{sbrk})~;
	\item \fname{networkCore}~: cœur de gestion du réseau~;
	\item \fname{pair}~: équivalent à \lstcpp{std::pair}~;
	\item \fname{process}~: gestion des processus et des SVC~;
	\item \fname{queue}~: équivalent à \lstcpp{std::queue}~;
	\item \fname{sleep}~: gestion des \lstc{sleep}, et plus généralement
		du temps (timers, \ldots)~;
	\item \fname{start\_message}~: contient une chaîne de caractères envoyée
		sur le réseau au boot~;
	\item \fname{startup}~: fonctions standard d'initialisation des programmes
		indépendants, inclus dans \fname{libsys.a}~;
	\item \fname{svc}~: appels SVC (appels système)~;
	\item \fname{syslib}~: header pour les programmes indépendants~;
	\item \fname{udp}~: gestion des paquets UDP~;
	\item \fname{udpSocket}~: socket (héritant de \lstc{GenericSocket})
		spécialisé pour l'UDP~;
	\item \fname{udpSysWrite}~: structure pour écrire de l'UDP sur le
		réseau depuis un programme indépendant~;
	\item \fname{uspi\_interface}~: interfaçage avec USPi (driver USB).
\end{itemize}

\subsection{Dépendances}

Pour fonctionner, le projet dans son état actuel nécessite~:
\begin{itemize}
	\item les outils de cross-compilation \texttt{arm-none-eabi} (comprenant
		\fname{g++}, \fname{gcc}, \fname{ld}, \fname{as}, \fname{ar},
		\fname{objdump}, \fname{objcopy} et autres)~;
	\item Python 3~;
	\item une Raspberry Pi B version 1.
\end{itemize}

\subsection{Compilation}
La compilation se déroule en plusieurs étapes~:
\begin{enumerate}
	\item compilation de \fname{libsys.a}~;
	\item compilation de chaque programme indépendant, décoration de ce
		fichier comme exécutable pour notre noyau et copie dans le
		dossier du RAMFS initial~;
	\item génération du fichier C++ définissant le contenu du RAMFS initial~;
	\item compilation des objets qui ne sont pas encore compilés pour le
		noyau~;
	\item assemblage dans \fname{\_build/output.elf}~;
	\item copie de l'objet compilé avec les décorations nécessaires en
		\fname{kernel.img}~;
	\item éventuellement (\lstbash{make install}), copie sur la carte SD\@.
\end{enumerate}

L'étape d'assemblage utilise un script de linker essentiellement repris de
\fname{arm-none-eabi-ld}, à une ou deux modifications près. Chaque
compilation vers un objet nécessite un certain nombre d'options, afin de
compiler pour la bonne architecture et sans le code d'initialisation, entre
autres.

La compilation des programmes indépendants est un peu compliquée également~:
comme nous n'avons pas utilisé de MMU, nous avons dû produire de l'assembleur
\textit{position-independant}, stockant ses variables globales à des endroits
adaptés (et non en plein milieu de la pile système, comme ce qui est fait
par défaut\ldots), etc.

%%%%%%%%%%%%%%%%%%%%%%%%%%%%%%%%%%%%%%%%%%%%%%%%%%%%%%%%%%%%%%%%%%%%%%%%%%%%%%%
\section{Bas niveau et matériel}

\subsection{Matériel}
Notre noyau a été conçu et testé sur une 
\href{https://www.raspberrypi.org/products/model-b/}{Raspberry Pi B version 1}.
Nous avons de plus soudé sur une plaque de prototypage quelques headers de
pins auxquels sont reliées huit LEDs identiques pour afficher un octet,
une LED d'état et une LED de crash, connectables par GPIO à la Raspberry Pi~;
ainsi qu'un interrupteur d'alimentation et les ports USB allant avec, afin
d'éviter d'user les connecteurs à brancher et débrancher le câble USB
d'alimentation (seule manière prévue d'allumer et d'éteindre la Raspberry Pi).

Le chargement du noyau se faisait en écrivant directement sur la carte SD
(qui sert de stockage et de disque de démarrage à la Raspberry Pi) à place
du \fname{kernel.img} installé par la distribution standard Linux pour
Raspberry Pi, à l'aide de \lstbash{make install}.

\subsection{Séquence de boot}
Nous avons conservé le bootloader par défaut de
\href{https://www.raspberrypi.org/downloads/raspbian/}{Raspbian} pour démarrer
notre \fname{kernel.img}. La séquence de boot standard (que nous n'avons donc
pas implémentée) démarre tout d'abord le GPU, qui va exécuter le bootloader,
charger (par simple \lstc{memcpy}) l'image système en RAM, puis démarrer le
CPU et lui donner la main.

Celui-ci déplace son \lstc{PC} (Program Counter) à l'adresse \hex{8000},
où doit se trouver la fonction de démarrage du noyau, correspondant pour nous
au contenu de \fname{init.s}, qui déplace le \lstc{SP} à \hex{8000} (la stack
croît vers le bas), puis saute sur le \lstc{\_c\_init} (dans \fname{cinit.cpp})
qui écrit le \lstc{BSS} et donne la main à \lstc{kernel\_main}.

\subsection{GPIO}
Les pins GPIO se contrôlent aisément à l'aide de constantes à écrire à un
endroit précis en mémoire pour en gérer le sens (input ou output) et en écrire
l'état.

\subsection{Mailbox}
Ce protocole permet d'interagir avec les différents composants matériels,
comme le GPU (non-implémenté), mais aussi la récupération de l'adresse MAC
de la carte réseau, la mise sous tension des composants, les capteurs de
température, \ldots

%%%%%%%%%%%%%%%%%%%%%%%%%%%%%%%%%%%%%%%%%%%%%%%%%%%%%%%%%%%%%%%%%%%%%%%%%%%%%%%
\section{Processus}

Pour gérer plusieurs processus s'exécutant en parallèle, on effectue
de la simulation séquentielle du parallélisme, en utilisant un seul
cœur du Raspberry Pi, qui exécute séquentiellement des morceaux de
chaque processus.

Pour faire la transition entre chaque processus, il s'agit de
sauvegarder son état (\ie{} l'état de tous ses registres et de son mode
uniquement, puisqu'il n'y a pas de MMU). On garde ces informations,
ainsi que quelques autres (état du processus, nom, répertoire
courant\ldots) dans une table globale.

Les appels système se font par l'intermédiaire des intructions SVC,
qui font un appel au superviseur (et en particulier, changent le mode
du processeur~; c'est la seule manière de sortir du mode
\textit{user}). Il y a un type d'appel SVC pour chaque appel système
souhaité~; chacun d'entre eux ayant du code associé dans
\fname{process.cpp}.

Les informations retenues pour chaque processus sont~:
\begin{itemize}
\item{l'état des registres et le mode,}
\item{les processus suivants et précédents dans la liste circulaire
  doublement chaînée des processus,}
\item{le répertoire courant,}
\item{l'état du processus~: actif, en train d'attendre de
  l'entrée-sortie, d'attendre la terminaison d'un autre processus,
  de dormir, ou zombi, ainsi qu'un entier 64 bits indiquant une
  information d'état (le moment de réveil pour un processus endormi,
  par exemple).}
\end{itemize}

Tout cela permet d'avoir des processus s'exécutant en même temps, et
pouvant communiquer à travers les sockets existants. Cependant, il
y a une difficulté pour exécuter des processus depuis un fichier.
\\

En effet, exécuter un processus depuis un fichier requiert le
chargement du fichier en mémoire, et l'exécution de celui-ci. Or, nous
n'avons pas de MMU, le fichier ne peut donc pas être facilement placé
à n'importe quel endroit de la mémoire. La solution pour cela est de
compiler les programmes avec l'option \texttt{-fPIC}, qui génère du
code indépendant de la position.

Cependant, ce code indépendant de la position a besoin d'une table
globale indiquant les positions des données. Pour ce faire, nous
ajoutons le décalage convenable aux variables dans la section
\texttt{.got} (Global Offset Table), ce qui permet d'avoir les
décalages voulus. Le format de l'en-tête d'un fichier exécutable est
donc le suivant~:
\begin{itemize}
\item{constante magique \lstc{"\\x7fELF"} (4 octets),}
\item{fin de la section \texttt{.bss} (4 octets), permet de savoir la
  mémoire statique à allouer au processus,}
\item{position de la section \texttt{.got} (4 octets),}
\item{longueur (en mots de 4 octets) de la section \texttt{.got} (4 octets).\\}
\end{itemize}

Malgré cela, il semblerait qu'il reste quelques problèmes avec la
relocalisation des exécutables~: ainsi, les exceptions et
l'utilisation de chaînes de caractères globales semblent poser
problème. Comme nous n'avions pas énormément de temps, et que le
format des fichiers produits par \lstbash{gcc} est complexe, nous
n'avons pas cherché à corriger ces problèmes (ce qu'il faudrait bien
sûr faire dans un vrai système d'exploitation~!).

%%%%%%%%%%%%%%%%%%%%%%%%%%%%%%%%%%%%%%%%%%%%%%%%%%%%%%%%%%%%%%%%%%%%%%%%%%%%%%%
\section{Réseau}
Notre projet de noyau s'appuie fortement sur le réseau, puisque nous n'avons
aucune interface (écran, clavier, souris, \ldots) branchée directement sur la
Raspberry Pi. Nous l'utilisons entre autres pour envoyer des logs
à d'autres machines et ouvrir un shell distant.

\subsection{Protocoles supportés}
Nous avons implémenté différents protocoles réseau afin d'aboutir à une
architecture représentant un bon compromis de simplicité d'utilisation
et d'implémentation.
\begin{itemize}
	\item Couche 2 (liaison de données)~: protocole \textbf{Ethernet}, en partie
		implémenté par USPi (enrobage des frames). Il restait à coder
		l'ajout d'adresses MAC (destination, origine) et un EtherType à la
		frame.
	\item Couche 3 (réseau)~: protocole \textbf{IPv4} servant d'enrobage à
		d'autres protocoles.
	\item Couche 3 (réseau)~: protocole \textbf{ARP} afin de découvrir les
		adresses MAC sur le réseau local (nécessaire pour le protocole
		Ethernet si on ne souhaite pas hardcoder les adresses MAC des
		machines).
	\item Couche 4 (transport)~: protocole \textbf{ICMP} partiellement
		implémenté pour répondre au ping.
	\item Couche 4 (transport)~: protocole \textbf{UDP} fortement utilisé
		pour toute la transmission de données. UDP était bien plus simple à
		implémenter que TCP, et semble fiable lorsqu'utilisé sur un réseau
		local (aucune perte de paquets notée pendant les tests).
\end{itemize}

Nous avons de plus implémenté un protocole ad-hoc pour le logger.
Lors de son initialisation, le logger envoie sur le réseau un paquet spécial
(sur couche UDP), reconnu par le client de logs. Celui-ci a alors 500\ ms pour
répondre un autre paquet spécial, afin que le logger l'enregistre comme
client. Le logger envoie ensuite ses logs à tous les clients ayant demandé à
les recevoir, que ce soit pendant la phase d'initialisation, ou plus tard
(mais alors, le client aura raté tous les logs de démarrage).

%%%%%%%%%%%%%%%%%%%%%%%%%%%%%%%%%%%%%%%%%%%%%%%%%%%%%%%%%%%%%%%%%%%%%%%%%%%%%%%
\section{Programmes indépendants}

\subsection{Shell et démon de shell réseau}
Nous avons implémenté un shell (\fname{ush}) et un démon de shell réseau
(\fname{ushd}). Ceci nous permet de nous connecter au port 22 de la
Raspberry Pi en UDP et d'obtenir une console. Celle-ci étant totalement
non-sécurisée, nous l'avons nommée Unsecure SHell (ou Unreliable SHell,
au choix)~: la connexion passe par UDP, n'est pas chiffrée, se fait sans mot
de passe, \ldots

Le démon écoute les paquets réseau sur le port 22 et ouvre des \fname{ush}
pour chaque nouveau client (\ie{} nouveau couple adresse-port), dont il gère
l'entrée et la sortie standard pour les interfacer avec le réseau.

\subsection{Commandes usuelles de shell}
Les commandes suivantes sont implémentées~:
\lstc{cat}, \lstc{echo}, \lstc{ls}, \lstc{ps}, \lstc{pwd}, \lstc{cd}.
%lstc et non lstbash pour éviter la coloration syntaxique.

\subsection{Interpréteur de Z-machine}
La Z-machine est une machine virtuelle inventée en 1979 et spécialisée
pour les jeux de fiction interactive~: elle a été créée dans le but de
rendre portables ce type de jeux publiés à cette époque, le nombre
d'architectures existant étant très élevé~! Cette objectif de
portabilité nous a permis d'en implanter également une version, qui
peut éxécuter de manière minimaliste les fichiers de la version 5 de
la Z-machine -- une des plus utilisées de nos jours. Ainsi, ce petit
programme ($\approx 1500$ lignes) nous permet de jouer à un grand
nombre de jeux (probablement plusieurs milliers~!).

L'interpréteur se contente de lire le fichier donné en argument, et
d'éxécuter son contenu en suivant la spécification
(c.f. \href{http://inform-fiction.org/zmachine/standards/z1point1/index.html}
{http://inform-fiction.org/zmachine/standards/z1point1/index.html})
lorsque c'était possible et pas trop compliqué, ou en se débrouillant
pour écrire quelque chose de minimaliste permettant aux jeux de
fonctionner dans les autres cas. Le lecteur est invité à essayer des
jeux comme
\href{http://ifdb.tads.org/viewgame?id=x6ne0bbd2oqm6h3a}{\textit{Balances}} ou
\href{http://ifdb.tads.org/viewgame?id=5e23lnq25gon9tp3}{\textit{All
    Things Devours}} pour se représenter les jeux en question~; une
partie importante des titres sortis par Infocom (dont \textit{Trinity}, un des
jeux du genre jugé parmi les meilleurs) devraient pouvoir être utilisés
avec l'interpréteur fourni.

\subsection{Serveur de Snake multijoueur}
Nous avons également implanté un petit jeu de Snake multijoueur en
Python, puis recodé le serveur en C++ en utilisant les appels système
de notre noyau afin de pouvoir éxécuter celui-ci sur le Raspberry
Pi. Il est ainsi possible de se connecter avec plusieurs joueurs,
chacun contrôlant un serpent indépendamment.

%%%%%%%%%%%%%%%%%%%%%%%%%%%%%%%%%%%%%%%%%%%%%%%%%%%%%%%%%%%%%%%%%%%%%%%%%%%%%%%
\section{Difficultés rencontrées}

% \subsection* : économiser de la place dans la ToC.

\subsection*{Recoder la bibliothèque standard}
Nous avons eu des problèmes systématiquement dès lors que nous avons essayé
d'utiliser les bibliothèques standard C et C++, qu'il s'agisse de problèmes de
lien des objets, d'utiliser notre \lstc{malloc} et non le \lstc{malloc}
standard, \ldots

Nous nous sommes donc résolus à recoder les parties de ces bibliothèques dont
nous avons eu besoin~: \lstc{printf}, quelques fonctions sur les chaînes de
caractères C, un tableau redimensionnable, une table de hachage, \ldots

\subsection*{Écrasement de la table de vecteurs}
La table de vecteurs se trouve à l'adresse mémoire \hex{00}. Une fonction de
mailbox nécessitait un bloc de mémoire alloué sur le tas aligné sur 4 octets,
et une inattention avait conduit à la ligne
\begin{lstlisting}[language=C++]
buff = (uint32_t*) (mem + ((16 - ((Ptr)mem & 0xFFFFFFF0)) & (0xFFFFFFF0)));
\end{lstlisting}

au lieu de
\begin{lstlisting}[language=C++]
buff = (uint32_t*) (mem + ((16 - ((Ptr)mem & 0xF)) & (0xFFFFFFF0)));
\end{lstlisting}

Le résultat est que \lstc{buff} se trouvait modulé --- et non aligné --- sur 4
octets, et toute écriture dans \lstc{buff} provoquait donc l'écrasement de la
table de vecteurs, entraînant une erreur incompréhensible\ldots surtout
lorsqu'on doit la débugger avec des LEDs.


\subsection*{Liaison C et C++ à la fois de malloc}
Pour faire marcher USPi, il est nécessaire d'implémenter les fonctions
définies dans le header \fname{uspi/include/uspios.h}, dont \lstc{malloc},
définie dans un bloc \lstcpp{extern "C"} dans ce header. % chktex 18

Or, \lstc{malloc} était également définie dans \fname{malloc.h}, sans liaison
C cette fois. Certains fichiers incluaient \fname{uspios.h} sans inclure
\fname{malloc.h}, ainsi aucun warning de redéfinition n'était levé~; toutefois,
si on appelait \lstc{malloc} depuis ces fichiers, les conventions d'appel
différentes provoquaient une erreur brutale, et tout aussi difficile à
trouver que la précédente.

\subsection*{\lstc{.text.startup} n'est pas au début}
Pour s'assurer que l'assembleur contenu dans \fname{init.s} serait placé par
le linker script à l'adresse \hex{8000}, nous avions manuellement indiqué
que celui-ci devait se trouver dans la section \lstc{.text.startup}.

Cela marchait très bien, jusqu'au jour où un constructeur jamais appelé a été
ajouté à une structure. Le compilateur a alors jugé bon de le placer dans
la section \lstc{.text.unlikely}, placée \emph{avant} \lstc{.text.startup}.

Ainsi, l'ajout d'un constructeur à une classe empêchait le démarrage
(pas le moindre clignottement de LED~!) de la Raspberry Pi, puisque le code
exécuté au démarrage (adresse \hex{8000}) était\ldots un constructeur d'une
structure inutile, puis un return.

\subsection*{Pile du mode SVC mal placée}
La gestion des interrupts se fait par copie des registres du processus
actuel sur la pile, afin de les sauver. Le code lançant le premier
processus mettait ainsi le pointeur de pile, qui n'était plus utilisé
par la suite au début de la table contenant les registres du processus
à lancer, par souci de cohérence avec le reste du code.

Cependant, lors de l'ajout des appels SVC, le pointeur de pile se
retrouvait alors là où on l'avait laissé, le Raspberry Pi démarrant en
mode SVC. La pile écrasait alors la table des processus, et provoquant
ainsi un crash difficile à expliquer.

\end{document}

